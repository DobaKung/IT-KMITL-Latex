\chapter{บทนำ}
\label{chapter:introduction}

\section{ที่มาและความสำคัญ}

การ์ตูนญี่ปุ่นเป็นที่รู้จักกันอย่างแพร่หลายทั่วโลกในฐานะสื่อบันเทิง หรืออีกชื่อหนึ่งคือ “มังงะ (Manga)” ในปัจจุบันมีงานวิจัยในหัวข้อมังงะอย่างหลากหลาย ในหลาย ๆ งานวิจัย~\cite{8369633, Liu2016, Pang2014, 7415523, ogawa2018, 7351614, 7452668} มีการใช้ชุดข้อมูลสำหรับการทดลอง เช่น Manga109~\cite{Matsui2017} ซึ่งเป็นชุดข้อมูลที่ถูกสร้างขึ้นจากภาพมังงะจำนวน 20,260 หน้า รวบรวมจากมังงะ 109 เรื่อง มังงะที่ถูกรวบรวมมานี้เป็นผลงานของนักวาดมังงะมืออาชีพชาวญี่ปุ่น นอกจากภาพของมังงะแล้ว ชุดข้อมูลนี้ยังประกอบไปด้วยข้อมูลอธิบายประกอบ หรือ Annotation ต่าง ๆ เช่น ขอบเขตและตำแหน่งของใบหน้า ร่ายกาย และ กรอบภาพ เป็นต้น นอกจากนี้ยังมีข้อมูลขอบเขตและตำแหน่งของข้อความที่ปรากฎในภาพมังงะ โดยตำแหน่งข้อความต่าง ๆ นั้นถูกป้อนข้อมูลด้วยแรงงานคนโดยไม่พึ่งพาระบบอัตโนมัติใด ๆ ในการป้อนข้อมูลดังกล่าวนั้นใช้เวลานานและต้องพึ่งพาแรงงานมนุษย์ ด้วยเหตุนี้ระบบอัตโนมัติที่จะสามารถเข้ามาช่วยในการระบุข้อมูล Annotation นั้นจึงมีประโยชน์และสามารถช่วยลดภาระงานในส่วนนี้ลงได้อย่างมาก

ถึงแม้ว่าสำหรับภาพวาดรูปแบบการ์ตูนญี่ปุ่นจะมีทั้งแบบภาพวาดทั่วไปที่เป็นภาพแสดงของตัวละครหรือทิวทัศน์และแบบภาพมังงะ แต่ภายในงานวิจัยนี้เรามุ่งเน้นไปที่มังงะเป็นหลักเนื่องจากข้อความมักปรากฎบนหนังสือการ์ตูนมากกว่าภาพวาดทั่วไปอย่างที่ทราบกันดี สำหรับวิธีการตรวจหาข้อความในภาพมังงะนั้นมีการพัฒนามาหลากหลายก่อนหน้านี้~\cite{6761596, 7490104} แต่วิธีเหล่านี้ถูกพัฒนาให้พึ่งพาโครงสร้างส่วนต่าง ๆ ของภาพมังงะเป็นข้อมูลอ้างอิง เช่น กรอบช่องภาพวาด, ลักษณะของกล่องคำพูด เป็นต้น นอกจากนี้บางวิธียังคงมีความจำเป็นที่ต้องพึ่งพาการป้อนข้อมูลเข้าจากภายนอกทั้งจากมนุษย์และข้อมูลอื่น ๆ ทำให้ไม่สามารถทำงานได้อัตโนมัติอย่างสมบูรณ์ อย่างไรก็ดีไม่นานมานี้มีการพัฒนาวิธีการใหม่โดยใช้วิธีการ Deep Learning อย่างเช่นเทคนิค Convolutional Neural Network เพื่อช่วยในการสกัดลักษณะเด่น (Feature) ออกจากภาพมังงะเพื่อช่วยในการตรวจหาข้อความในภาพมังงะ~\cite{7532890} ซึ่งวิธีการนี้สามารถเพิ่มความแม่นยำและถูกต้องในการตรวจหาข้อความได้โดยปราศจากการพึ่งพาโครงสร้างต่าง ๆ ในภาพมังงะ แต่อย่างไรก็ดี Deep Learning ยังเป็นการวิธีการที่ต้องใช้ทรัพยากรของระบบเพื่อการคำนวนมากกว่าวิธีการอื่น ๆ ซึ่งเป็นข้อเสียสำคัญประการหนึ่ง~\cite{7532890}

ในงานวิจัยนี้มีจุดมุ่งหมายเพื่อพัฒนาระบบตรวจหาข้อความที่ทำงานได้กับมังงะอย่างหลากหลายและไม่ถูกจำกัดด้วยโครงสร้างหรือลักษณะบางประการของภาพมังงะ เราจึงเลือกใช้ Stroke Width Transform (SWT) ในการสกัดลักษณะเด่นของเส้นต่าง ๆ ของวัตถุที่ปรากฎในภาพออกมา โดยวิธีการนี้ถูกใช้เป็นขั้นตอนแรกของการตรวจหาข้อความบนภาพถ่ายมาก่อนหน้านี้ วิธีการนี้ทำงานโดยพึ่งพาสมมติฐานว่าขอบของเส้นอักษรในข้อความนั้นมีขอบที่ชัดเจนและหนาแน่นปรากฏอยู่บนพื้นหลังที่ราบเรียบ~\cite{5540041} อย่างไรก็ดีการใช้วิธีการนี้กับการตรวจหาข้อความบนภาพมังงะส่งผลให้เกิดข้อผิดพลาดเชิง False Positive จำนวนมาก ปัญหานี้เกิดจากความแตกต่างของลักษณะเฉพาะตัวของภาพถ่ายและภาพวาดมังงะ ภาพมังงะนั้นโดยส่วนใหญ่มีลักษณะเป็นภาพขาวดำ และลักษณะของวัตถุภายในมังงะ เช่น ขนาด, เส้น, และพื้นหลัง นั้นมีความคล้ายคลึงกับตัวอักษรของข้อความ ด้วยปัญหาข้างต้นเราจึงตั้งเป้าหมายในการปรับปรุงและพัฒนา SWT ที่ถูกใช้ในภาพถ่าย~\cite{5540041} เพื่อให้สามารถทำงานกับภาพมังงะได้

วิธีการใหม่ของเราที่ถูกพัฒนาขึ้นใหม่นั้นแบ่งออกเป็น 4 ส่วนดังนี้ (\rNum{1}) The Stroke Width Transform (\rNum{2}) ค้นหาวัตถุที่เข้าข่ายลักษณะของตัวอักษร (\rNum{3}) คัดแยกอักษร โดยใช้ Support Vector Machine (SVM) ร่วมกับ Histogram of Oriented Gradients Feature (\rNum{4}) จัดกลุ่มอักษรที่ผ่านการคัดแยกแล้วให้เกิดเป็นบรรทัดหรือกลุ่มของข้อความ

\section{วัตถุประสงค์}
พัฒนาระบบค้นหาตำแหน่งข้อความสำหรับมังงะ โดยนำ Stroke Width Transform ที่ถูกใช้เป็นกระบวนการแรกเริ่มในเทคนิคตรวจหาข้อความบนภาพถ่ายมาพัฒนาและปรับปรุงต่อยอดเพื่อให้สามารถใช้งานกับภาพมังงะได้มีประสิทธิภาพมากขึ้น

\section{ขอบเขตของงานวิจัย}
\begin{enumerate}
    \item พัฒนาระบบตรวจหาตำแหน่งข้อความซึ่งใช้สำหรับภาพมังงะ โทนสีขาวดำ
    \item ภาษาของเนื้อหาในมังงะที่นำมาใช้งาน คือ ภาษาญี่ปุ่น
    \item ข้อมูลที่ใช้ในการวิจัยเพื่อการเทรนและทดสอบนำมาจากฐานข้อมูล Manga109
\end{enumerate}

\section{ประโยชน์ที่คาดว่าจะได้รับ}
\begin{enumerate}
    \item ได้วิธีการตรวจหาข้อความใหม่ที่ถูกพัฒนาขึ้นเพื่อใช้งานร่วมกับภาพมังงะโดยเฉพาะ
    \item ทำให้ทราบถึงลักษณะที่เป็นเอกลักษณ์ของมังงะซึ่งแตกต่างจากภาพถ่ายทั่วไป
\end{enumerate}


